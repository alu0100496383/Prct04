\documentclass[a4paper,12pt]{article}
\usepackage[utf8]{inputenc}
\usepackage{graphicx}
\usepackage[spanish]{babel}
\begin{document}
\title{Fórmulas matemáticas en LaTeX}
\author{Mérari Afonso \\
        Técnicas Experimentales~\footnote{Universidad de la Laguna}
        }
\date{\today}
\maketitle
\begin{abstract}
  Una de las grandes ventajas de \LaTeX{}~\cite{Lam:86} es la 
  existencia de una gran cantidad de "paquetes" estándares pensados 
  para dotar a los textos de toda la funcionalidad que se precise. 
  Así hay paquetes para incluir gráficos, textos de lenguajes de programación,
  fórmulas físicas y químicas, diagramas matemáticos, etc.
  
  Por ejemplo:  \[h^2=a^2 + b^2 \]
\end{abstract}

\section{Ambientes matemáticos para LaTeX}
Como anteriormente se mencionó \LaTeX{}~\cite{Lam:86} es especialmente apropiado 
para imprimir fórmulas, ecuaciones y símbolos matemáticos. 
Para esto existe el ambiente matemático, el cual se puede utilizar 
a través de 4 comandos que son: math (para fórmulas en el texto), 
displaymath (para una línea de ecuación no numerada), equation 
(para una línea de ecuación numerada) y por último eqnarray 
(para varias líneas de ecuación). $\\par$.

\subsection{Ambiente Math}
Este ambiente se utiliza para intercalar fórmulas en las líneas de texto, 
por ejemplo si se escribe: El teorema de Pitágoras, $x^{2} + y^{2} = h^{2}$ . 
\subsection{Ambiente displaymath y equation}
Este comando se utiliza para imprimir una ecuación en el centro de la línea. Los comandos son:
Si encierras la fórmula entre los comandos begin{displaymath} y end{displaymath}, 
ocasiona que la fórmula no sea numerada.
En cambio si se utiliza begin{equation} y end{equation} la fórmula será numerada. Y cada 
vez que ingreses una nueva fórmula en este ambiente se incrementará automáticamente el número.
Se escribía asi:
\begin{verbatim}
\begin{displaymath}
               x^{2} + y^{2} = h^{2}
\end{displaymath}
\end{verbatim}
\begin{verbatim}
\begin{equation}
	      x^{2} + y^{2} = h^{2}
\end{equation} 
\end{verbatim}
El resultado en \LaTeX{}~\cite{Lam:86} sería:
\begin{displaymath}
               x^{2} + y^{2} = h^{2}
\end{displaymath}
\begin{equation}
	      x^{2} + y^{2} = h^{2}
\end{equation} 

\subsection{Ambiente eqnarray}
Existe un cuarto modo que es el modo eqnarray. Este modo está pensado para escribir ecuaciones 
multilineas o ecuaciones que exceden al ancho de linea; se comporta como una matriz de tres columnas 
donde la primera alinea a derecha, la segunda al centro y la tercera a la izquierda.

Las ecuaciones que queramos presentar en este modo deben encerrarse entre begin{eqnarray} y end{eqnarray}; si quisiéramos 
las mismas sin numeración podemos utilizar eqnarray* en vez de eqnarray.
Se escribiría así:
\begin{verbatim}
\begin{eqnarray*}
 \frac1{t - z} & = & \frac1{t - a - (z - a)} \\
 & = & \frac1{t - a}
  \left( \frac1{1 - \frac{z - a}{t - a}} \right) \\
 & = & \frac1{t - a}
  \left[ \sum_{i=0}^n \left( \frac{z - a}{t - a} \right)^n
  + \frac{\left( \frac{z - a}{t - a} \right)^{n+1}}
  {1 + \frac{z - a}{t - a}} \right]
\end{eqnarray*}
\end{verbatim}
El resultado en \LaTeX{}~\cite{Lam:86} sería:
\begin{eqnarray*}
 \frac1{t - z} & = & \frac1{t - a - (z - a)} \\
 & = & \frac1{t - a}
  \left( \frac1{1 - \frac{z - a}{t - a}} \right) \\
 & = & \frac1{t - a}
  \left[ \sum_{i=0}^n \left( \frac{z - a}{t - a} \right)^n
  + \frac{\left( \frac{z - a}{t - a} \right)^{n+1}}
  {1 + \frac{z - a}{t - a}} \right]
\end{eqnarray*}

\section{Referencia de ecuaciones}
Es posible referenciar una ecuación desde cualquier lugar del texto, para ello se utilizan dos comandos, estos son
\begin{verbatim}
\label{key} (En al ecuación)

~\ref{key} (Insertada en el lugar del texto donde la ecuación es citada)

\end{verbatim}
Cualquier texto puede ser usado para reemplazar a key. Suponga que tiene la siguiente ecuación con el primer comando
\begin{verbatim}
          \begin{equation}
               Y = 4x^{2} - 3x + 5  \label{ecua}
          \end{equation}
\end{verbatim}
En cualquier lugar del texto donde escriba \begin{verbatim} ~\ref{ecua} \end{verbatim} obtendrá la ecuación.
Ejemplo:
\begin{equation}
               Y = 4x^{2} - 3x + 5  \label{ecua}
          \end{equation}
Yo quiero que la rerefencia a esa ecuación salga justamente aquí: ~\ref{ecua}

Nota : Al igual que los otros comandos para referenciar páginas o secciones este comando requiere 
que el documento sea compilado dos veces. 

\bigskip
\begin{tabular}{|l|c|c|c|}
\hline
Nombre & Día & Mes & Año \\ \hline
Mérari & 21 & 07 & 1990 \\ \hline
Nacho & 29 & 07 & 1990 \\ \hline
\end{tabular}
\includegraphics[width=0.5\textwidth]{imagen1.eps}
\end{document}


