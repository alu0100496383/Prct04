\documentclass[a4paper,12pt]{article}
\usepackage[utf8]{inputenc}
\usepackage{graphicx}
\usepackage[spanish]{babel}
\begin{document}
\title{Fórmulas matemáticas en LaTeX}
\author{Mérari Afonso \\
        Técnicas Experimentales~\footnote{Universidad de la Laguna}
        }
\date{\today}
\maketitle
\begin{abstract}
  Una de las grandes ventajas de \LaTeX{}~\cite{Lam:86} es la 
  existencia de una gran cantidad de "paquetes" estándares pensados 
  para dotar a los textos de toda la funcionalidad que se precise. 
  Así hay paquetes para incluir gráficos, textos de lenguajes de programación,
  fórmulas físicas y químicas, diagramas matemáticos, etc.
  
  Por ejemplo:  \[h^2=a^2 + b^2 \]
\end{abstract}

\end{document}
